\chapter{Results}
\label{results}

%intro to the results

%iterations of results - DIP, HIPPIE, Bayesian updating and reasoning behind it
% can then refer to this throughout this section.

%Feature extraction results
\section{PPI feature vectors}

%The features extracted were X,Y,Z and the appendices explaining how this was done are A,B,C).

%Larger features
%Gene ontology was built as a feature in the same manner as that of qi_evaluation_2006 but, without knowing their approach, we had to develop our own method of creating usable features.

%ENTS features were retreived through analysis and modification of the code published on the ENTS website, but did not have full coverage on any dataset.

%Before their removal the features that best predicted the chosen gold standard dataset were reliably those directly derived from interaction databases.

%example feature importance graph

%After removing interaction databases 

%graph of feature importance from RF classifier?

%tables with explanation

\subsection{Data visualisation}

\subsubsection{Reducing dimensionality}
%PCA is a relatively simply and fast way to reduce the high dimensionality of our feature vector into a form that can be easily plotted.


\subsection{High dimensional plots}

%Very few graphs are able to integrate large numbers of dimensions in a meaningful way; parallel line graphs and Andrew's curves are the two we have applied.

%Explain how Andrew's curves work, what they mean.


\subsection{Missing data}
%how was missing data dealt with?

%Justification for mean value filling.

\section{Classification of weighted PPI networks}

%Accuracy as a simple measure of the performance of a classifier is difficult to interpret in the case of a heavily unbalanced classifier such as this.
\subsection{Classifier accuracy and best parameters}

%Using grid searches over the following parameter ranges we were able to search for the optimal parameters for each of the classifiers tested.

%table of the best parameters obtained for each classifier.

\subsection{ROC curves}

%An ROC curve plots the tradeoff between true positive and false positive rates, in the case of an unbalanced classifier large sample sizes are required to obtain a smooth, stable curve.

%ROC curves for the different classifiers

%differences between the classifiers and reasons for this.

\subsection{Precision-recall curves}

%precision recall curves for the different classifiers

\subsection{Feature importances}

%Comparing logistic regression to random forests

%tests characterising and comparing different classifiers, results

%Treating interactions as an unobserved random variable, we were able to build a simple probabilistic model to make up for the failings of the classifier and continue with the Community Detection.

\section{Comparison of weighted and unweighted PPI networks}

%Here are the communities we detected in each case

%images of both sets of communities, nicely rendered

%Investigate some of the communities by eye, look at distribution of baits etc

\subsection{Graph comparison}

%comparison of using weighted and unweighted
%NMI and disease enrichment

\section*{Conclusion}


