\chapter{Results}
\label{results}

%intro to the results

%iterations of results - DIP, HIPPIE, Bayesian updating and reasoning behind it
% can then refer to this throughout this section.
As the project progressed the intended approach was found to be flawed and some changes were made.
The first of these was the change from a DIP-based training set to HIPPIE-based training set.
After the classification had been performed the use of a supervised classifier with this training set was found to be a poor method by itself for weighting interactions.
Using all of the data available it was possible to run a simple Bayesian alternative to continue with the weighting for the Community detection algorithm.

%Feature extraction results
\section{PPI feature vectors}

%The features extracted were X,Y,Z and the appendices explaining how this was done are A,B,C.
Many features were identified for extraction and these are described in Appendix \ref{datasources}.
Of these, only a small subset were succesfully processed into a usable form.
These are listed below and more information about each can be found in Appendix \ref{datasources}:

%more citations below?
\begin{itemize}
    \item HIPPIE database
    \item Pulldown derived features: affinity and abundance
    \item Gene Ontology, also described in section \ref{go}
    \item Yeast Two-Hybrid, also described in section \ref{y2h}
    \item ENTS derived features, also described in section \ref{ents}
    \item iRefIndex database
    \item STRING database
    \item HMR database 
    \item InterologWalk results
\end{itemize}

Of these, a smaller proportion were used in the final classifier, which neglected features directly derived from interaction databases.
The features used to train the final classifier are shown in table \ref{tab:features}.
A brief description of these features is given below.

%feature table, with information about each (size of features, categorical/ordinal/numerical, coverage)
\begin{table}
    \centering
    %tabular goes here
    \begin{tabular}{l c c c c}
        Feature         &   Size &  Type                &  Coverage on training set &  Coverage on active zone network \\
        \hline
        Gene Ontology    &  90   &  Binary categorical  &  100.0\%                  & 100.0\%                          \\
        Yeast two-hybrid &  1    &  Numerical           &  100.0\%                  & 100.0\%                          \\
        ENTS derived     &  107  &  Numerical           &  38.39\%                  & 42.74\%                          \\
    \end{tabular}
    \caption{A table summarising the components of the feature vectors used in the final classifier.}
    \label{tab:features}
\end{table}

%these subsections may be too small
\subsection{Gene Ontology}
\label{go}

%Larger features
%Gene ontology was built as a feature in the same manner as that of qi_evaluation_2006 but, without knowing their approach, we had to develop our own method of creating usable features.
The Gene Ontology\autocite{ashburner_gene_2000} is a resource of annotations for genes to indicate various characteristics in a hierarchical manner, such as cellular localisation or function.
This resource has been used in past papers\autocite{qi_evaluation_2006} and in databases such as STRING\autocite{von-mering_string:_2005} to predict protein interactions.
Intuitively, it can be used to detect when, for example, two proteins are localised in the same area of the cell - as this would increase the probability that these two proteins interact.
Details on exactly how this feature was generated can be found in the notebook reference in Appendix \ref{app:go}.

\subsection{Features derived from ENTS}
\label{ents}
%ENTS features were retreived through analysis and modification of the code published on the ENTS website, but did not have full coverage on any dataset.

These features were obtained through analysis and modification of the bundled code and data downloaded from the website of \textcite{rodgers-melnick_predicting_2013}.
In turn, most of these features were generated through the Multiloc2 program of \textcite{blum_multiloc2:_2009}.
The remaining features are pairwise combinations of conserved protein domains, which are conserved "modules" of proteins described in \textcite{janin_domains_1985}.

\subsection{Yeast Two-Hybrid results}
\label{y2h}

%description of Y2H feature
%pending

\subsection{Removed features}

%Before their removal the features that best predicted the chosen gold standard dataset were reliably those directly derived from interaction databases.
As listed above there were originally many features used in the supervised classification that were derived from interaction databases.
These were very effective in predicting interactions on the training set, as expected, but their importance in the task outweighed any other features, as shown in figure \ref{fig:unbalanced}.
It was decided that indirect features should be used in the trained supervised classifier and direct evidence integrated into the final weightings in an explicit Bayesian method described in section \ref{bayes}; the results of which are described in section \ref{bayesresults}.

%example feature importance graph
\begin{figure}
    \centering
    \includegraphics[width=0.8\textwidth]{unbalanced.weighting.png}
    \caption{An example of an unbalanced set of feature importances plotted after fitting a Random Forest classifier to a dataset containing interaction database derived features.}
    \label{fig:unbalanced}
\end{figure}

%After removing interaction databases 
Once these databases were removed the performance of the classifier was drastically lower.
However, all of the available features had more closely distributed importances in the final classifier, as shown in section \ref{importances}.
The classifier was then integrated 

%graph of feature importance from RF classifier? probably not required, will be found below.

%tables with explanation - table above now, not required.

\subsection{Data visualisation}

%describe the problem of using in proportion and out of proportion methods
For all graphs, two cases were investigated: in proportion and out of proportion.
In proportion refers to the case where the proportion of interactions to non-interactions is correct; specifically, 1 interaction to 600 non-interactions.
Out of proportion maintains the same number of interactions to non-interactions.
This is important as it is often easier to separated the data when the classes are equally split.

\subsubsection{Reducing dimensionality}
%why do we want to reduce the dimensionality
If the data were two dimensional we would like to plot it as a scatter plot and see if there was a clear grouping of the points.
This would indicate that a classification algorithm would be able classify the data.
As the data has in excess of one hundred dimensions it is necessary to reduce the dimensionality before plotting.

%PCA is a relatively simply and fast way to reduce the high dimensionality of our feature vector into a form that can be easily plotted.
Two methods were tested to reduce the dimensionality of the data to two dimensions so that it could be easily plotted.
The first of these is PCA, which is relatively simple with a fast implementation, and the second is t-SNE, which is more complicated but has achieved better performance in recent works.
Both of these methods are described below in more detail.

%PCA description, referencing Barber
In PCA the method to express a high-dimensional point $\pmb{x}$ as a low dimensional point relies on finding an approximation for this point according to\autocite[330]{barber_bayesian_2013}:

\begin{align}
    \pmb{x}^{n} \approx \pmb{c} + \sum_{j=1}^{M} y_{j}^{n} \pmb{b}^{j} \equiv \tilde{\pmb{x}}^{n}
\end{align}

Where the low dimensional points are $\pmb{y}^{n}$, $\pmb{c}$ is a constant and $\pmb{b}_{j}$ are the principal components.
The algorithm to find these principal components is described in \textcite[333]{barber_bayesian_2013}.

The resulting plot produced using this technique is shown in figures \ref{fig:inpca} and \ref{fig:outpca} for the in proportion and out cases, respectively.
In the case of the data being out of proportion it appears that the groups can be separated.
However, this is not the case for the in proportion case.

%plots
\begin{figure}
    \setlength\figureheight{3in}
    \setlength\figurewidth{4in}
    \InputIfFileExists{in.pca.tikz}{}{\textbf{!! Missing graphics !!}}
    \centering
    \caption{In proportion plot of the data reduced to two dimensions using PCA.}
    \label{fig:inpca}
\end{figure}

\begin{figure}
    \setlength\figureheight{3in}
    \setlength\figurewidth{4in}
    \InputIfFileExists{out.pca.tikz}{}{\textbf{!! Missing graphics !!}}
    \centering
    \caption{Out of proportion plot of the data reduced to two dimensions using PCA.}
    \label{fig:outpca}
\end{figure}


%tSNE is more complicated, but was recommended due to reportedly good performance
A more complex method to reduce the dimensionality of the data is t-SNE\autocite{van_der_maaten_visualizing_2008}.
This technique won the Merck Visualization Challenge.
It differs from PCA in the objective function in that it aims to maintain a similarity measure between points between the high and low-dimensional cases.

%should probably say what we did vis. TruncatedSVD
In addition to using this technique it was also recommended to use Truncated Singular Value Decomposition(SVD) to reduce the dimensionality of the vector to 50 beforehand.
This was for implementation based reasons in Scikit-learn.

%results
Unfortunately, this was no more successful than PCA on this dataset, as shown in figures \ref{fig:intsne} and \ref{fig:outtsne}.
It should also be noted that the number of points plotted in these graphs is much lower than in the case of PCA as it is much more computationally intensive.

%plots
\begin{figure}
    \setlength\figureheight{3in}
    \setlength\figurewidth{4in}
    \InputIfFileExists{in.tsne.tikz}{}{\textbf{!! Missing graphics !!}}
    \centering
    \caption{In proportion plot of the data reduced to two dimensions using t-SNE.}
    \label{fig:intsne}
\end{figure}

\begin{figure}
    \setlength\figureheight{3in}
    \setlength\figurewidth{4in}
    \InputIfFileExists{out.tsne.tikz}{}{\textbf{!! Missing graphics !!}}
    \centering
    \caption{Out of proportion plot of the data reduced to two dimensions using t-SNE.}
    \label{fig:outtsne}
\end{figure}

%Both methods show that this data is likely to be difficult to accurately categorize as the points are not separated in a 2d space
Both of the methods above suggest that this classification problem is difficult, as the points are not significantly separated in any of the plots.

\subsection{High dimensional plots}

%Very few graphs are able to integrate large numbers of dimensions in a meaningful way; parallel line graphs and Andrew's curves are the two we have applied.
Neglecting reducing the dimensionality of the data there are some plots which are able to illustrate a very large number of dimensions.
Two which have been applied in this project are parallel lines plots and Andrew's curves\autocite{andrews_plots_1972}.
Parallel lines plots are relatively simple in that each feature is simply scaled and plotted along the x axis at its index in the vector, producing a number of overlapping lines.
Andrew's curves are more complex, and the method is described below along with the results obtained.

%parallel line plot results
The parallel lines plots are shown in proportion in figure \ref{fig:inparline} and out of proportion in figure \ref{fig:outparline}.
In either case, it is not clear whether the data easily separates into two classes.
The in proportion case in particular shown in figure \ref{fig:inparlines} is very difficult to separate into interactions and non-interactions.

%this should be a tikz plot
\begin{figure}
    \setlength\figureheight{3in}
    \setlength\figurewidth{4in}
    \InputIfFileExists{in.parallel.lines.tikz}{}{\textbf{!! Missing graphics !!}}
    \centering
    \caption{In proportion parallel line plot of the data used to train the final classifier.}
    \label{fig:inparline}
\end{figure}

%this should be a tikz plot
\begin{figure}
    \setlength\figureheight{3in}
    \setlength\figurewidth{4in}
    \InputIfFileExists{out.parallel.lines.tikz}{}{\textbf{!! Missing graphics !!}}
    \centering
    \caption{Out of proportion parallel line plot of the data used to train the final classifier.}
    \label{fig:outparline}
\end{figure}

%Explain how Andrew's curves work, what they mean.
Andrew's curves are similar to parallel line plots in that they consist of overlapping lines.
The lines in Andrew's curves are generated by mixing waves at different frequencies weighted based on the values of the features in the vector\autocite{andrews_plots_1972}.
The result is a periodic wave that is modulated based on the values of the feature vectors.
If the feature vectors of each class have reliably different values this should be reflected in the grouping of these curves.
The two cases of in proportion and out are shown in figures \ref{fig:inandcurve} and \ref{fig:outandcurve}, respectively.

\begin{figure}
    \setlength\figureheight{3in}
    \setlength\figurewidth{4in}
    \InputIfFileExists{in.andrews.curves.tikz}{}{\textbf{!! Missing graphics !!}}
    \centering
    \caption{In proportion Andrew's curves plot of the data used to train the final classifier.}
    \label{fig:inandcurve}
\end{figure}

\begin{figure}
    \setlength\figureheight{3in}
    \setlength\figurewidth{4in}
    \InputIfFileExists{out.andrews.curves.tikz}{}{\textbf{!! Missing graphics !!}}
    \centering
    \caption{Out of proportion Andrew's curves plot of the data used to train the final classifier.}
    \label{fig:outandcurve}
\end{figure}

%what is shown in the plots?
In neither graph is it easy to see a grouping of the curves being plotted.
This means that it is likely to be very difficult to accurately classify the vectors.

\section{Classification in weighted PPI networks}

%why is this different to regular classification
In a classification task there are two common components: a training set that is labeled and a set of data which the classifier is to be applied to, without labels.
This task is similar in that we have created a training set with labels and the active zone network forms a set of feature vectors that have no labels.
However, this does not accurately encode our prior knowledge of the active zone network interactions, which were picked for their likelihood to be true interactions.
For this reason, in this application classification alone is not sufficient to solve the problem of creating accurate weights, driving the development of the technique described in section \ref{bayes}.

This section describes both the results of training the classifier as planned on the labeled training set and the results of the Bayesian method to weight the interactions of the active zone network.

%Accuracy as a simple measure of the performance of a classifier is difficult to interpret in the case of a heavily unbalanced classifier such as this.
\subsection{Classifier accuracy and best parameters}

%Using grid searches over the following parameter ranges we were able to search for the optimal parameters for each of the classifiers tested.
Grid searches of hyper-parameter values were used to find the optimal combination.
The results of this search are shown in table \ref{tab:gridresults}.

%table of the best parameters obtained for each classifier.
\begin{table}
    \centering
    \begin{tabular}{l c c c c}
        Classifier                                  & Hyper-parameter values    & Test set accuracy         &                               & Training set accuracy         &  \\
        \hline
        Logistic Regression                         & C : $0.0215$              & $0.9984$                  & $\pm 3\e{-5}$                 & $0.99847$                     & $\pm 2\e{-5}$ \\
        \multirow{3}{*}{Support Vector Machine}     & kernel: RBF               & \multirow{3}{*}{$0.9985$} & \multirow{3}{*}{N/A}          & \multirow{3}{*}{$0.99933$}    & \multirow{3}{*}{$\pm 1.2\e{-4}$} \\
                                                    & Gamma: $10.0$             &                           &                               &                               &       \\
                                                    & C: $10.0$                 &                           &                               &                               &       \\
        \multirow{2}{*}{Random Forest}              & N estimators: $44$        & \multirow{2}{*}{$0.99837$} & \multirow{2}{*}{$\pm 3\e{-5}$} & \multirow{2}{*}{$0.99921$}   & \multirow{2}{*}{$\pm 3\e{-5}$} \\
                                                    & Max features: $25$        &                           &                               &                               &       \\
        \multirow{2}{*}{Extremely Randomized Trees} & N estimators: $94$        & \multirow{2}{*}{$0.99837$} & \multirow{2}{*}{$\pm 3\e{-5}$} & \multirow{2}{*}{$0.99921$}   & \multirow{2}{*}{$\pm 3\e{-5}$} \\
                                                    & Max features: $25$        &                           &                               &                               &       \\
    \end{tabular}
    \caption{Summary of the hyper-parameter combinations found by grid search and associated statistical accuracy measures.}
    \label{tab:gridresults}
\end{table}

Although the results of the Support Vector Machine appear to be good, it was trained on a relatively small dataset and its performance in later tests deemed it unsuitable to our classification task.
The random forest can be seen to be performing worse than the logistic regression model in this test, with a slightly lower accuracy, but within the standard error boundaries.

\subsection{ROC curves}

%An ROC curve plots the tradeoff between true positive and false positive rates, in the case of an unbalanced classifier large sample sizes are required to obtain a smooth, stable curve.


%ROC curves for the different classifiers

%differences between the classifiers and reasons for this.

\subsection{Precision-recall curves}

%what a precision recall plot is?

%precision recall curves for the different classifiers


\subsection{Feature importances}
\label{importances}

%Comparing logistic regression to random forests

%tests characterising and comparing different classifiers, results

%Treating interactions as an unobserved random variable, we were able to build a simple probabilistic model to make up for the failings of the classifier and continue with the Community Detection.

\subsection{Bayesian weighting of interactions}
\label{bayesresults}

%description of the Bayesian method of interaction weighting, with reference to the notebook on this

\section{Comparison of weighted and unweighted PPI networks}

%Here are the communities we detected in each case

%images of both sets of communities, nicely rendered

%Investigate some of the communities by eye, look at distribution of baits etc

\subsection{Graph comparison}

%comparison of using weighted and unweighted
%NMI and disease enrichment

\section*{Conclusion}


