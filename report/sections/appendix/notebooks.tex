\appendix
\chapter{Notebooks}

This project took advantage of IPython notebooks to run interactive scripting for the diverse tasks necessary to organise the data.
Additionally, this allowed the project to maintain comprehensive records of work done.
It would be possible for anyone with access to the data to run this code again to verify it.
The code was run with Python version 2.7.7 and Scikit-learn 0.15.0.

\section{Classifier Training}
\label{app:classtrain}

This notebook contains all the code that was run to train and test the classifier used in this project.
It involves model selection, grid search of parameters and various plots describing the performance of different classifiers, such as ROC curves and Precision Recall curves.
Links are provided to the code and an online service to view the notebook:

\begin{itemize}
    \item The notebook in the opencast-bio repository can be found here: \url{https://github.com/ggray1729/opencast-bio/blob/master/notebooks/Classifier%20Training%20HIPPIE.ipynb}
    \item A viewable version of this notebook can be found here: \url{http://nbviewer.ipython.org/github/ggray1729/opencast-bio/blob/master/notebooks/Classifier%20Training%20HIPPIE.ipynb}
\end{itemize}



\lipsum[10-20]


