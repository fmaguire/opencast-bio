\chapter{Notebooks}
\label{app:notebooks}

The job of quickly running arbitrary processing on a variety of different data sources, each of which are being encountered for the first time was approached using IPython notebooks.
Quick interactive programming was useful as unexpected problems could be quickly solved.
Also, a detailed log, with inline comments, could be kept to track exactly what was done.

It would be possible for anyone with access to the data to run this code again to verify it.
The code was run with Python version 2.7.7 and Scikit-learn 0.15.0.
The notebooks can be found at the following locations:

%link to root directory
\begin{itemize}
    \item The root directory for these notebooks can be found here: \url{https://github.com/ggray1729/opencast-bio/tree/master/notebooks}
    \item These can be viewed here: \url{http://nbviewer.ipython.org/github/ggray1729/opencast-bio/tree/master/notebooks/}
\end{itemize}

\section{Feature Extraction Notebooks}

Of the feature extraction notebooks in the repository, not all of them were successful.
Only those that were successful are listed here.

\subsection{Gene Ontology}
\label{app:go}

In total 90 features were extracted from the Gene Ontology as binary values.
The following notebook describes how the features applied were generated:

\begin{itemize}
    \item The notebook in the opencast-bio repository can be found here: \url{https://github.com/ggray1729/opencast-bio/blob/master/notebooks/Extracting%20Gene%20Ontology%20features%200.2.ipynb}
        \item This notebook can be viewed here: \url{http://nbviewer.ipython.org/github/ggray1729/opencast-bio/blob/master/notebooks/Extracting%20Gene%20Ontology%20features%200.2.ipynb}
\end{itemize}

\subsection{Features derived from ENTS}

%pending
107 features were generated derived from the work in \textcite{rodgers-melnick_predicting_2013}.
These were found by analysing the code provided on the papers web page as described in the following notebook:

\begin{itemize}
    \item The notebook in the opencast-bio repository can be found here: \url{https://github.com/ggray1729/opencast-bio/blob/master/notebooks/Inspecting%20ENTS%20code.ipynb}
        \item This notebook can be viewed here: \url{http://nbviewer.ipython.org/github/ggray1729/opencast-bio/blob/master/notebooks/Inspecting%20ENTS%20code.ipynb}
\end{itemize}

\section{Classifier Training}
\label{app:classtrain}

This notebook contains all the code that was run to train and test the classifier used in this project.
It involves model selection, grid search of parameters and various plots describing the performance of different classifiers, such as ROC curves and Precision Recall curves.
Links are provided to the code and an online service to view the notebook:

\begin{itemize}
    \item The notebook in the opencast-bio repository can be found here: \url{https://github.com/ggray1729/opencast-bio/blob/master/notebooks/Classifier%20Training%20HIPPIE.ipynb}
    \item This notebook can be viewed here: \url{http://nbviewer.ipython.org/github/ggray1729/opencast-bio/blob/master/notebooks/Classifier%20Training%20HIPPIE.ipynb}
\end{itemize}

