\chapter{Methodology}
\label{methods}

%intro to the methods
\lipsum[5]

\section{Feature extraction}

\lipsum[10-15]

\subsection{IPython notebooks}

The job of quickly running arbitrary processing on a variety of different data sources, each of which are being encountered for the first time was approached using IPython notebooks.
Quick interactive programming was useful as unexpected problems could be quickly solved.
Also, a detailed log, with inline comments, could be kept to track exactly what was done.

\subsection{Protein identifier mapping}

Mapping from one protein identifier to another became a significant problem in this project.
Unfortunately, most Biological databases maintain their own indexing method to identify different genes and proteins.
New data sources being integrated into this project would often be using a different identification scheme to that originally chosen to use in PPI network work at Edinburgh: the NCBI Entrez identifier.

%what the Entrez identifier is

%methods used during the project, with references to appendix notebooks

% talk about the problem of canonicalisation

% how using Entrez identifiers risks becoming gene interaction prediction
% or "What Entrez isn't"

% how iRefIndex solves this problem and should have been used from the start
% with reference to storing the sequences of each protein involved to maintain unambiguity


%description of feature extraction code
\subsection{Dedicated code: ocbio.extract}

\lipsum[15-20]

% link to the notebook on using this code
% but update it to explain what custom generator options are

\subsection{Gold standard datasets}
%material on the problems with choosing between gold-standard datasets

\lipsum[21-25]

%original work on DIP (justified choice from previous work)

%problems with DIP

%why HIPPIE is better suited

\subsection{PPI prediction features}

%this section should be written with reference to which features were found in the end to be important



\subsection{Parallel processing with IPython.parallel}
%description of how this was set up and how it could scale

\lipsum[26-28]

%details of what classifiers were chosen and about scikit-learning
\section{Weighting Protein Interactions}

\lipsum[1-10]

\subsection{Supervised binary classification problems}

% going through posing the problem in terms of probability

\subsection{Classification algorithms}

% explain why we tried the algorithms that we tried

% what other algorithms could we have tried but didn't

\subsubsection{Beta regression}

%why this would also have been a good idea, if we didn't have to implement it ourselves

%community detection code, what it does, which algorithm was used
\section{Measures applied to weighted and unweighted PPI networks}

\lipsum[15-20]

%NMI and disease enrichment methods

\section*{Conclusion}

\lipsum[1]
