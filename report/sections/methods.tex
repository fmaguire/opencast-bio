\chapter{Methodology}
\label{methods}

%intro to the methods
\lipsum[5]

\section{Feature extraction}

Features are the processed form of the data we use to make predictions about a given interaction between pairs of proteins and is defined mathematically in section \ref{supbinclass}.
The process of taking a retrieving these from biological databases is referred to as feature extraction.
This commonly involved mapping between protein identifiers and indexing large tables automatically, but also involved many other small pieces of scripting.

\subsection{IPython notebooks}

The job of quickly running arbitrary processing on a variety of different data sources, each of which are being encountered for the first time was approached using IPython notebooks.
Quick interactive programming was useful as unexpected problems could be quickly solved.
Also, a detailed log, with inline comments, could be kept to track exactly what was done.

%list of all the feature extraction ipython notebooks, with references to the appendices

\subsection{Supervised binary classification}
\label{supbinclass}

% going through posing the problem in terms of probability
In supervised learning problems we wish to learn a mapping between input variables and output variables given a training set.
Defining this training set rigorously, it consists of input variables $\mathbf{x}$ which are typically vectors of values known as features.
The output variables in a classification problems are a set of labels\cite[2]{murphy_machine_2012}.
In the case of binary classification these are simply either 0 or 1.
Therefore, given $N$ training vectors $\mathbf{x}_{i}$ and training labels $y_{i}$ we can define our training set $\mathcal{D}$ as:

\begin{align}
    \mathcal{D} = \left{ ( \mathbf{x}_{i}, y_{i} ) \right}_{i=1}^{N}
\end{align}

%an example?

%pose our problem
Our problem involves taking various types of biological data, such as entries from biological databases indicating that proteins are involved in the same part of a cell and using these as features.
The training labels are either an interaction (a one) or a non-interaction (a zero).
Interactions are taken to be any interactions in the HIPPIE\cite{schaefer_hippie:_2012} database with over 50\% confidence.
Negative interactions are three million random binary combinations of Entrez protein IDs, which is a method applied in other works\cite{qi_evaluation_2006} to create negative training examples.
There are many more negative that positive interactions and it is unlikely that any of the negative examples are real interactions.

What we would like to estimate is the posterior probability of an interaction existing given a new feature vector after training our classifier.
For a model $\mathcal{H}$ and a new feature vector $\mathbf{x}^{*}$ we can express this using Bayes theorem:

\begin{align}
    p(y^{*} = 1 | \mathbf{x}^{*}, \mathcal{D}, \mathcal{H}) = \frac{ p(\mathbf{x}^{*}| y^{*} = 1 , \mathcal{D}, \mathcal{H}) p( y^{*} = 1 | \mathcal{D}, \mathcal{H})}{\sum{y^{*}} p( \mathbf{x}^{*} | y^{*}, \mathcal{D}, \mathcal{H})p( y^{*} = 1 | \mathcal{D}, \mathcal{H})}
\end{align}

These expressions are defined:

\begin{itemize}
    \item The posterior probability:
        \begin{align}
            p(y^{*} = 1 | \mathbf{x}^{*}, \mathcal{D}, \mathcal{H})
        \end{align}
    \item The likelihood:
        \begin{align}
            p(\mathbf{x}^{*}| y^{*} = 1 , \mathcal{D}, \mathcal{H})
        \end{align}
    \item The prior:
        \begin{align}
            p( y^{*} = 1 | \mathcal{D}, \mathcal{H})
        \end{align}
    \item The marginal likelihood:
        \begin{align}
            \sum{y^{*}} p( \mathbf{x}^{*} | y^{*}, \mathcal{D}, \mathcal{H})p( y^{*} = 1 | \mathcal{D}, \mathcal{H})
        \end{align}

We do not apply explicitly apply a prior to the probability of interaction, meaning that we implicitly apply a uniform prior.

\subsection{Protein identifier mapping}

Mapping from one protein identifier to another became a significant problem in this project.
Unfortunately, most Biological databases maintain their own indexing method to identify different genes and proteins.
New data sources being integrated into this project would often be using a different identification scheme to that originally chosen to use in PPI network work at Edinburgh: the NCBI Entrez identifier.

%what the Entrez identifier is -  as opposed to other protein identifier schemes - cite NCBI web pages
Genes are defined by their amino acid sequence, but this is a long series of letters and the number of genes is much smaller than the possible combinations of these letters.
For the sake of posterity databases containing information about genes typically apply an identifier for each gene that is much shorter and can encode other information about the gene.
The Entrez GeneID identifier is relatively simple, just consisting of a number generated when the gene was added to the database\cite{maglott_entrez_2006}.

%other protein identifier schemes and mapping between them
Other popular schemes include the Ensembl identifier from the Ensembl database\cite{ensembl_website}, Uniprot identifiers from the Uniprot database\cite{uniprot_website} and even those used only for specific databases such as DIP identifiers\cite{dip_website}.
Mapping between these different identifiers is difficult as each identifier may map to none or many in another database.
The reason this happens is due to isoforms of different proteins; different amino acid sequences can code for a protein with the same name.


%methods used during the project, with references to appendix notebooks
Various tools exist to map from one protein identifier to another:

\begin{itemize}
    \item Ensembl's BioMart\cite{smedley_biomart_2009}:
        \begin{itemize}
            \item Able to map from various identifiers to others based on different databases.
            \item Produces a csv which can be used in a variety of different tools.
            \item Requires a list of protein identifiers.
        \end{itemize}
    \item NCBI's Gene\cite{maglott_entrez_2006} provides conversion tables on it's ftp servers:
        \begin{itemize}
            \item Simple tab-separated text file converting all known GeneIDs between different formats.
        \end{itemize}
    \item The Uniprot\cite{consortium_universal_2007} online service:
        \begin{itemize}
            \item Similar to BioMart, but with a simpler interface.
            \item Converts only to or from Uniprot identifiers.
            \item Requires a list of protein identifiers.
        \end{itemize}
\end{itemize}

% talk about the problem of canonicalisation
Unfortunately, using any of these services there will be a number of IDs which cannot be converted and many IDs mapping to the same Entrez ID as different protein isoforms are picked up.
One way to avoid this problem is to only refer to a single canonical form of any given protein and find this protein in other databases through its amino acid sequence.
This ensures that when referring to an interaction between two identifiers the interaction is always simply between two proteins.

% how using Entrez identifiers risks becoming gene interaction prediction
% or "What Entrez isn't"
Otherwise, as in this project, the interaction is detected between two Entrez IDs; which corresponds to an interaction between genes - possibly only a single interaction between combinations of the isoforms of each gene.
Unfortunately, this means that this project is only concerned with gene interaction prediction until the Entrez IDs have been carefully canonicalized.
This is not really a problem, as we are only aiming to provide a weighting to a graph, rather than provide an accurate prediction of interaction between proteins.

% how iRefIndex solves this problem and should have been used from the start
% with reference to storing the sequences of each protein involved to maintain unambiguity
A solution to this problem is provided by the iRefIndex\cite{razick_irefindex_2008} database, which combines many databases and stores canonicalized entries.
Using this database, it would be possible to ensure that the proteins used in a future project would be reliable canonical proteins.
Additionally, each protein of interest should ideally be stored with reference to its sequence in, for example, FASTA format.

%description of feature extraction code
\subsection{Dedicated code: ocbio.extract}

\lipsum[15-20]

% link to the notebook on using this code
% but update it to explain what custom generator options are

\subsection{Gold standard datasets}
%material on the problems with choosing between gold-standard datasets
% with reference here to the section in Qi's thesis
\lipsum[21-25]

%original work on DIP (justified choice from previous work)
The database of interacting proteins (DIP) is a database of interactions proven by small-scale experiments\cite{xenarios_dip_2002}.
Each interaction added is hand curated, so it should be fairly reliable as a training set.

%problems with DIP

%why HIPPIE is better suited

%problem with positive vs. negative ratio with HIPPIE
% and how it's not a problem

\subsection{PPI prediction features}

%this section should be written with reference to which features were found in the end to be important

%a graph showing the importance of different features is required

%a table describing each feature used, and where it comes from

\subsection{Parallel processing with IPython.parallel}
%description of how this was set up and how it could scale

To take maximum advantage of the available computing facilities and because the sample sizes in the project exceed one million the decision was made to prepare the code for parallel processing on a remote server.
Particularly, grid search operations to optimize performance of the classifier were considered to be processor intensive and vital to the success of the prediction task.
The easiest way to set up these interactive parallel processing operations was the parallel processing model in IPython\cite{parallel_python_webpage}.

%How this worked in practice, and the potential
The notebooks using parallel processing are the notebooks on classifier training, which are described in Appendix \ref{app:classtrain}.
This usage depended on code from a parallel processing tutorial\cite{ogrisel_parallel} to distribute memory to the cores using Numpy's memmap methods.
The code to do this has been integrated into the ocbio module in the project repository and can be used as shown in the notebooks.
Potentially, and as described in the tutorial, this code could be used to run the classifier training on cloud services using Starcluster.

%how it should probably have been done differently
However, the gains from parallel processing were mitigated by trying to run the notebook server on the remote server.
The clusters could have been initialised and the jobs submitted locally to the remote server operating as a cluster, thus taking advantage of all the available resources.
This would also be a better system if the parallel processing were to be scaled up to use other processing clusters.

%details of what classifiers were chosen and about scikit-learning
\section{Weighting Protein Interactions}

%classification as a weighting tool
The classification problem we are solving is different in that we are really trying to obtain a realistic weighting of interactions for use in a PPI network.
Classification is normally concerned about picking a decision threshold to classify examples into categories.
However, in this case the output of our process is the posterior probability of the model given a new example.

%introduce scikit-learn
To produce this output the chosen tool was the Python package Scikit-learn\cite{pedregosa_scikit-learn:_2011}.
Each classifier implemented in this package has a similar interface allowing modular code to be written.
In addition, this package is actively developed with all the required classifiers having efficient implementations.

\subsection{Classification algorithms}

% explain why we tried the algorithms that we tried
Three classification algorithms were chosen to test on the data:

\begin{itemize}
    \item Logistic regression
    \item Support vector machine (SVM)
    \item Random forest
\end{itemize}

The reasons for choosing these and brief description of each algorithm are given below.

\subsubsection{Logistic Regression}

%with reference to Murphy

\subsubsection{Support Vector Machines}

%with reference to Murphy

\subsubsection{Random Forest}

%with reference to Qi and ENTS, showning good performance on this problem

\subsubsection*{Other options}

% what other algorithms could we have tried but didn't
Other options considered for our classification problem, but not included in the project due to time constraints included:

\begin{itemize}
    \item Feedforward neural networks
    \item Naive Bayes
    \item Beta regression
\end{itemize}

Naive Bayes in particular would have required modifying the code from Scikit-learn to deal with data from multiple different distributions or implementing Weka's solution of kernel density estimated distributions for each different feature\cite{john_estimating_1995}.
Beta regression would have been very suitable for the task and is suggested as future work, described in the following section.

\subsubsection{Beta regression}

%why this would also have been a good idea, if we didn't have to implement it ourselves

\subsubsection{Classifier verification}

%Tests we planned to use on each classifier

\section{Measures applied to weighted and unweighted PPI networks}

%why do we want to try weighted networks?

\subsection{Community detection}

%community detection code, what it does, which algorithm was used

\subsection{Normalised Mutual Information}

%NMI theory, what is a measure of

\subsection{Disease Enrichment}

%Primer on what this test actually does

\section*{Conclusion}

\lipsum[1]
